
\documentclass[12pt]{article}

\usepackage{physor2016}

\usepackage{amsmath}
\usepackage{bm}

\newcommand{\ve}[1]{\ensuremath{\mathbf{#1}}}
\newcommand{\Macro}{\ensuremath{\Sigma}}
\newcommand{\vOmega}{\ensuremath{\hat{\Omega}}}
\newcommand{\omvec}{\ensuremath{\hat{\Omega}}}
\newcommand{\rvec}{\ensuremath{\vec{r}}}
\newcommand{\vecr}{\ensuremath{\vec{r}}}

\usepackage{graphicx}
\usepackage{tikz}
\usepgflibrary{shapes.geometric}
\usepackage{hyperref}
\hypersetup{
    colorlinks,
    citecolor=black,
    filecolor=black,
    linkcolor=black,
    urlcolor=black
}

\usepackage{booktabs}

\usepackage{siunitx}
%------------------------------------------------------------------------------

%------------------------------------------------------------------------------
% Define title. Use all CAPITALS.
%------------------------------------------------------------------------------
\title{Improved Hybrid Modeling of Spent Fuel Storage Facilities}
%\subtitle{}
%
% ...and authors
%
\author{ 
  Year 1 Annual Report for Project 14-6378\\
  January 2015 to December 2015\\
  \\
  Prepared by:\\
  Rachel Slaybaugh (PI), Assistant Professor \\
  Department of Nuclear Engineering, University of California, Berkeley \\
  4173 Etcheverry Hall, Berkeley, CA 94720, USA\\
  +1 570 850 3385, 
  \href{mailto:slaybaugh@berkeley.edu}{slaybaugh@berkeley.edu}\\
  \\
  Thomas Evans, Scott Moser\\
  \\
  Performed Under: NEUP Award DE-NE0008286\\
  Project period: 01/01/15 - 12/31/17 \\
  \\
  TPOC: Piyush Sabharwall\\
  FPOC: Daniel Vega\\
  \\
  Submitted May 13, 2016
}
  
%------------------------------------------------------------------------------
  

%------------------------------------------------------------------------------
% Setup PDF info. This sets several values which are listed as the "properties"
% of the PDF file.
%------------------------------------------------------------------------------
\hypersetup{
  pdftitle=\shorttitle,
  pdfauthor=\shortauthor
}


\begin{document}

%\doublespacing

%\linenumbers

%------------------------------------------------------------------------------
% Make the titlepage and set the pagestyle to fancy throughout
%------------------------------------------------------------------------------
\maketitle
\clearpage
\tableofcontents
\clearpage
%\begin{abstract}
%A new method for generating variance reduction parameters for strongly anisotropic, deep-penetration radiation shielding studies is presented. This method generates an alternate form of the adjoint scalar flux quantity, $\phi^{\dagger}_{\Omega}$, which is used by both CADIS and FW-CADIS to generate variance reduction parameters for local and global response functions, respectively. The new method, called CADIS-$\Omega$, was implemented in the Denovo/ADVANTG software suite, and results are presented for a concrete labyrinth test problem. Results indicate that the flux generated by CADIS-$\Omega$ incorporates localized angular anisotropies in the flux effectively. CADIS-$\Omega$ outperformed CADIS in the test problem while obtaining correct results. This initial work indicates that CADIS-$\Omega$ may be highly useful for shielding problems with strong angular anisotropies. A future test plan to fully characterize the new method is proposed, which should reveal more about the types of realistic problems for which the CADIS-$\Omega$ will be suited. 
%\end{abstract}
%
%\keywords{Hybrid Methods, CADIS, FW-CADIS, Angular Biasing}

%------------------------------------------------------------------------------
%
%------------------------------------------------------------------------------
\section*{Executive Summary}
\label{sect::summary}

\clearpage
\section{Significant Developments}
\label{sect::sig-devel}
Developments that have a significant favorable impact on the project.

\section{Performance Comparison}
\label{sect::perf-comp} 
A written comparison of the actual project accomplishments with the project goals and objectives established for the reporting period; if goals and/or objectives for the reporting period were not met, a detailed description of the variance shall be provided. 

\section{Accomplishments}
\label{sect::accomplishments}
A discussion of what was accomplished under these goals and objectives established for this reporting period, including major activities, significant results, major findings or conclusions, key outcomes or other achievements.  This section should not contain any proprietary data or other information not subject to public release.  If such information is important to reporting progress, do not include the information, but include a note in the report advising the reader contact the Principal Investigator for further information. 

\section{Cost Status}
\label{sect::cost}
Cost Status.  A comparison of the approved budget by budget period and the actual costs incurred during the reporting period shall be provided.  If cost sharing is required, the cost breakdown shall show the DOE share, recipient share, and total costs. 

\section{Schedule Status}
\label{sect::schedule}
Schedule Status.  List milestones, anticipated completion dates and actual completion dates.  If you submitted a project management plan with your application, you must use this plan to report schedule and budget variances.   

\section{Changes}
\label{sect::changes}
Describe any changes during the reporting period in project approach and the reasons for these changes.  Remember, significant changes to the project objectives and scope require prior approval by the Contracting Officer. 

\section{Anticipated Issues}
\label{sect::schedule}
6.	Describe any actual or anticipated problems or delays and any actions taken or plan to resolve them 

\section{Personnel Changes}
\label{sect::personnel}
Describe any absence or changes of key personnel or changes in consortium/teaming arrangements during the reporting period. 

\section{Products}
\label{sect::products}
List and describe any product produced or technology transfer activities accomplished during this reporting period, such as: \\
a)	Publications (list journal name, volume, issue); conference papers; or other public releases of results.  Attach or send copies of the public releases to the DOE Program Manager. \\
b)	Web site or other Internet sites (list the URL) that reflect the results of this project. \\
c)	Networks or collaborations fostered. \\
d)	Technologies/Techniques (identify and describe each). \\
e)	Inventions/Patent Applications (identify and describe with date of application) \\
f)	Other products, such as data or databases, physical collections, audio or video, software or NetWare, models, educational aid or curricula, instruments or equipment (identify and describe).


\bibliographystyle{year1-annual-report}
\bibliography{year1-annual-report}

\appendix

\makeatletter
\def\@seccntformat#1{APPENDIX \csname the#1\endcsname.~}
\makeatother

%------------------------------------------------------------------------------
% If you need to make one (or more) appendix (appendices), place them here as
% sections
%%------------------------------------------------------------------------------
%\section{HOW TO MAKE APPENDICES}
%\label{app::a}
%
%This is a placeholder for my first appendix
%
%\section{OTHER APPENDIX STUFF}
%\label{app::b}
%
%This is a placeholder for my second appendix

\end{document}

