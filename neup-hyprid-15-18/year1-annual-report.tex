
\documentclass[12pt]{article}

\usepackage{physor2016}

\usepackage{amsmath}
\usepackage{bm}

\newcommand{\ve}[1]{\ensuremath{\mathbf{#1}}}
\newcommand{\Macro}{\ensuremath{\Sigma}}
\newcommand{\vOmega}{\ensuremath{\hat{\Omega}}}
\newcommand{\omvec}{\ensuremath{\hat{\Omega}}}
\newcommand{\rvec}{\ensuremath{\vec{r}}}
\newcommand{\vecr}{\ensuremath{\vec{r}}}

\pagestyle{myheadings}

\usepackage{graphicx}
\usepackage{tikz}
\usepgflibrary{shapes.geometric}
\usepackage{hyperref}
\hypersetup{
    colorlinks,
    citecolor=black,
    filecolor=black,
    linkcolor=black,
    urlcolor=black
}

\usepackage{booktabs}

\usepackage{siunitx}
%------------------------------------------------------------------------------

%------------------------------------------------------------------------------
% Define title. Use all CAPITALS.
%------------------------------------------------------------------------------
\title{Improved Hybrid Modeling of Spent Fuel Storage Facilities}
%\subtitle{}
%
% ...and authors
%
\author{ 
  Year 1 Annual Report for Project 14-6378\\
  January 2015 to December 2015\\
  \\
  \\
  Prepared by:\\
  Rachel Slaybaugh (PI), Assistant Professor \\
  Department of Nuclear Engineering, University of California, Berkeley \\
  4173 Etcheverry Hall, Berkeley, CA 94720, USA\\
  +1 570 850 3385, 
  \href{mailto:slaybaugh@berkeley.edu}{slaybaugh@berkeley.edu}\\
  \\
  Thomas Evans, Scott Moser\\
  \\
  \\
  Performed Under: NEUP Award DE-NE0008286\\
  Project period: 01/01/15 - 12/31/17 \\
  \\
  TPOC: Piyush Sabharwall\\
  FPOC: Daniel Vega\\
  \\
  \\
  Submitted May 13, 2016
}
  
%------------------------------------------------------------------------------
  

%------------------------------------------------------------------------------
% Setup PDF info. This sets several values which are listed as the "properties"
% of the PDF file.
%------------------------------------------------------------------------------
\hypersetup{
  pdftitle=\shorttitle,
  pdfauthor=\shortauthor
}


\begin{document}

%\doublespacing

%\linenumbers

%------------------------------------------------------------------------------
% Make the titlepage and set the pagestyle to fancy throughout
%------------------------------------------------------------------------------
\maketitle
\clearpage
\tableofcontents
\clearpage
%\begin{abstract}
%A new method for generating variance reduction parameters for strongly anisotropic, deep-penetration radiation shielding studies is presented. This method generates an alternate form of the adjoint scalar flux quantity, $\phi^{\dagger}_{\Omega}$, which is used by both CADIS and FW-CADIS to generate variance reduction parameters for local and global response functions, respectively. The new method, called CADIS-$\Omega$, was implemented in the Denovo/ADVANTG software suite, and results are presented for a concrete labyrinth test problem. Results indicate that the flux generated by CADIS-$\Omega$ incorporates localized angular anisotropies in the flux effectively. CADIS-$\Omega$ outperformed CADIS in the test problem while obtaining correct results. This initial work indicates that CADIS-$\Omega$ may be highly useful for shielding problems with strong angular anisotropies. A future test plan to fully characterize the new method is proposed, which should reveal more about the types of realistic problems for which the CADIS-$\Omega$ will be suited. 
%\end{abstract}
%
%\keywords{Hybrid Methods, CADIS, FW-CADIS, Angular Biasing}

%------------------------------------------------------------------------------
%
%------------------------------------------------------------------------------
\section*{Executive Summary}
\label{sect::summary}
\clearpage

\section{Project Description}
\label{sect::project}

In this project, we are developing a variance reduction method for computational neutral particle transport intended to improve the ability to design and operate monitoring systems for interim used fuel installations. 
We are using Exnihilo \cite{evans_denovo:_2010}, MCNP \cite{brown_mcnp_2002}, and ADVANTG \cite{mosher_new_2010}, which will make our new tool widely available and easily usable. 
These innovative analysis tools will enable next generation nuclear material management for existing and future U.S.\ nuclear fuel cycles, minimizing proliferation and terrorism risk.

Storage casks are particularly challenging for radiation transport calculations because they are characterized by dense shields followed by streaming paths to the detectors. 
The impact of streaming is amplified in arrays of casks because detectors can only see rear casks through air paths between front casks. 
Deterministic methods suffer from ray effects between the casks and detectors, making solutions unreliable. 
Monte Carlo methods are challenged by getting particles out of the cask into the region of interest, and can therefore require unreasonably large calculation times to achieve acceptable statistical uncertainties in the computed tallies.

\subsection{Current Technology}
Cutting edge variance reduction methods that speed up Monte Carlo calculations often use deterministic solutions to make weight window maps. 
Perhaps the most widespread and accessible of these methods are the Consistent Adjoint Driven Importance Sampling (CADIS) \cite{wagner_automatic_1997,wagner_automated_1998,haghighat_monte_2003} and Forward-Weighted CADIS (FW-CADIS) \cite{wagner_forward-weighted_2007,wagner_forward-weighted_2009,wagner_forward-weighted_2010} methods. 
These methods create consistently biased source distributions and weight window targets using a determinstic solution for the adjoint scalar flux, $\phi^{\dagger}$, as a measure of the importance. 
Equations \eqref{eq:CADISmethod} describe the biasing parameters generated by CADIS and FW-CADIS (jointly referred to as FW/CADIS). 

\begin{subequations} 
\label{eq:CADISmethod} 
\begin{equation}
\hat{q}(\vec {r} ,E)  = \frac{\phi^{\dagger}(\vec {r} ,E)q(\vec {r} ,E)}{\iint\phi^{\dagger}(\vec {r} ,E)q(\vec {r} ,E) dE d\vec{r}} \\
         = \frac{\phi^{\dagger}(\vec {r} ,E)q(\vec {r} ,E)}{R} \:,
\label{eq:weightedsource}
\end{equation}
\begin{equation}
w_0(\vec {r} ,E)  = \frac{q}{\hat{q}} \\
     = \frac{R}{\phi^{\dagger}(\vec {r} ,E)} \:,
\label{eq:startingweight}
\end{equation}
\begin{equation}
\hat{w}(\vec {r} ,E) = \frac{R}{\phi^{\dagger}(\vec {r} ,E)} \:,
\label{eq:WW}
\end{equation}
\end{subequations}
where $\hat{q}$ is the biased source distribution, $w_0$ is the starting weight of the particles, $\hat{w}$ is the target weight of the particles, and $R$ is the response of interest.

For problems with strong anisotropies in the particle flux, the importance map and biased source developed using the space/energy treatment above may not represent the real importance well enough to sufficiently improve performance in the Monte Carlo calculation. 
Note that because the \textit{scalar} adjoint flux is used in Eqns.~\eqref{eq:CADISmethod}, the angular dependence of the importance function is not retained. 
Thus, no information is retained on how particles move towards the response function. 
The drawback of this simplification is that, within a given space/energy cell, the map provides the average importance of a particle moving in \textit{any direction} through the cell\textemdash excluding information about how particles move \textit{toward} the objective. 
However, if the angular dependence of the importance function were fully retained, the map would be very large (tens or hundreds of GB) and more costly to use in the Monte Carlo simulation. 

Interim used fuel installations exhibit strong angular anisotropies, and therefore the ability to simulate them effectively for nuclear material management is limited with current tools. 

\subsection{New Method}
To do fast, accurate transport for used fuel monitoring, we need an importance map generated quickly using deterministic methods that captures the impact of angle in the importance information. 
In this work we build on past methods but calculate the adjoint scalar flux in a way that has not been done before.

Our new automated hybrid method, which we're calling FW/CADIS-$\Omega$, incorporates angular information into the biasing parameters for FW/CADIS while not explicitly biasing in angle. 
That is, we generate space- and energy-dependent importance maps that incorporate the flux anisotropy in a more effective way than current implementations without adding the complication of angular weight windows. 
FW/CADIS-$\Omega$ uses Eqns.~\eqref{eq:CADISmethod}, but generates the adjoint scalar flux differently. 

We use the idea of the contributon flux, defined in Eqn.~\eqref{eq.Cont-Flux}, in generating the adjoint scalar flux. 
Contributons are pseudo-particles that carry ``response" from the radiation source to a detector ~\cite{williams_generalized_1991,williams_contributorn_1992,williams_contributon_study}. 
%
\begin{equation}
\Psi (\vec {r},\:\hat\Omega ,E) = \psi^{\dagger} (\vec {r},\:\hat\Omega ,E) \psi(\vec {r} ,\:\hat\Omega,E)
\label{eq.Cont-Flux} 
\end{equation}
%
The contributon flux includes both forward and adjoint information, expressing the importance of a particle that is born at a forward source and moves through space towards and adjoint source, contributing to the solution.
An importance map made using contributon flux will assign high importance to particles that are created at the forward source and likely to generate a response in the detector. 

In FW/CADIS-$\Omega$, we integrate the contributon flux over angle and divide by the integrated forward angular flux as shown in Eqn.~\eqref{eq:angularhybrid}.
This quantity, which we designate $\phi^{\dagger}_{\Omega}$, is then used in Eqns.~\eqref{eq:CADISmethod}, just like the standard FW/CADIS methods.
%
\begin{equation} 
\phi^{\dagger}_{\Omega}(\vec{r},E) = \frac{\int_{4\pi} \psi(\vec {r} ,E,\hat{\Omega})\psi^{\dagger}(\vec {r} ,E,\hat{\Omega})d\hat\Omega }{\int_{4\pi}\psi(\vec {r} ,E,\hat{\Omega})d\hat\Omega}
\label{eq:angularhybrid}
\end{equation}

In a strongly anisotropic system, the adjoint scalar flux generated by Eqn.~\eqref{eq:angularhybrid} will be influenced by which directions were most prominent in the forward case. 
We can see this by considering the contributon flux, the numerator of Eqn.~\eqref{eq:angularhybrid}.
Particles in $\phi^{\dagger}_{\Omega}$ include the impact of how the direction they are moving influences the answer, 
which should allow for more effective Monte Carlo transport when angular effects are important. 
Note that in an isotropic system, $\phi^{\dagger}_{\Omega}$ will be essentially the same as $\phi^{\dagger}$. 

The UC Berkeley personnel working on this project are doctoral candidate Madicken Munk, undergraduate Garrett Baltz, and researcher Richard Vasques. We are being supported by Tom Evans and Scott Mosher at Oak Ridge National Laboratory (ORNL). 

\section{Significant Developments}
\label{sect::sig-devel}

As of December 31, 2016, there have been three main developments that have a significant favorable impact on the project:
\begin{enumerate}
\item Background research is completed\\
Completion of background research means that we have a full picture of past methods, current methods, and additional ideas to try if this method proves to perform less well than we would like or not well enough in certain situations.
This piece is also essential for inclusion in publications and Ms.\ Munk's dissertation.
 
\item Code implementation is nearly completed\\
Full implementation of the new method is required to conduct testing and impact studies. 
Significant progress here is key to project success.
Exnihilo, the deterministic code supplying adjoint flux values for weight windows, has been successfully modified to store and write adjoint flux (previously the only output was scalar flux). 
Further, an integration utility that implements Eqn.~\eqref{eq:angularhybrid} is also complete. 
The remaining piece is to automate the interface between the integration utility and ADVANTG, which will allow the full method to be used seamlessly using only ADVANTG input files. 

\item A large, representative test problem input specification is complete\\
The main item of interest is how this work will effect storage cask simulation. 
We have a full spent fuel storage cask model with overpack constructed in the input format required for ADVANTG. 
We started with a SCALE cask model and added an overpack, including rebar. 
This model was contributed back to ORNL's cask modeling library. 
Next, this model was converted to MCNP input syntax for use with ADVANTG.
We are now ready to start conducting impact calculations.
\end{enumerate}



\section{Performance Comparison}
\label{sect::perf-comp} 
A written comparison of the actual project accomplishments with the project goals and objectives established for the reporting period; if goals and/or objectives for the reporting period were not met, a detailed description of the variance shall be provided. 

We set the following goals for the project 


\section{Accomplishments}
\label{sect::accomplishments}
A discussion of what was accomplished under these goals and objectives established for this reporting period, including major activities, significant results, major findings or conclusions, key outcomes or other achievements.  This section should not contain any proprietary data or other information not subject to public release.  If such information is important to reporting progress, do not include the information, but include a note in the report advising the reader contact the Principal Investigator for further information. 

\section{Cost Status}
\label{sect::cost}
Cost Status.  A comparison of the approved budget by budget period and the actual costs incurred during the reporting period shall be provided.  If cost sharing is required, the cost breakdown shall show the DOE share, recipient share, and total costs. 


\begin{table}
\begin{center}
\caption{Proposed and Actual Costs for Year 1}
\begin{tabular}{ | l | l | l | l | l | }
\hline
	\textbf{Item} & \textbf{Per} & \textbf{Total} & \textbf{Actual Per} & \textbf{Actual Total }\\ \hline
	\textit{Section A, Senior/Key Person} & & \$11,632 & & \$4,845.12  \  \\ \hline
	\textit{Section B, Other Personnel} & &  \$96,777 & & \$78,504.19 \  \\ \hline
	Total Number Other Personnel & 4 & & \ 4 & \    \\ \hline
	\textit{Total Salary, Wages and Fringe Benefits (A+B)} & &  \$108,409 & & \$83,349.31 \  \\ \hline
	\textit{Section C, Equipment} &  &  \$-    & \ &  \$-     \\ \hline
	\textit{Section D, Travel} & &  \$8,000 & & \$2,438.24   \\ \hline
	1.  Domestic & \$8,000 & &  \$2,438.24 & \  \\ \hline
	2.  Foreign &  \$-  & \   &  \$-    & \   \\ \hline
	\textit{Section E, Participant/Trainee Support Costs} & \ &  \$-    & \  &  \$-       \\ \hline
	1.  Tuition/Fees/Health Insurance &   \$-  & \   &  \$-      \ & \\ \hline
	2.  Stipends &  \$-   & \ &  \$-    & \    \\ \hline
	3.  Travel &  \$-    & \ &  \$-    & \   \\ \hline
	4.  Subsistence &  \$-    & \ &  \$-     & \  \\ \hline
	5.  Other &  \$-    & \  &  \$-    & \  \\ \hline
	6.  Number of Participants/Trainees & 0 & \ & 0 & \   \\ \hline
	\textit{Section F, Other Direct Costs}& \ & \$45,000  & & \$42,255.41    \\ \hline
	1.  Materials and Supplies & \$500 & & \$455  & \  \\ \hline
	2.  Publication Costs & \$500  &  &  \$-   & \  \\ \hline
	3.  Consultant Services &  \$-    & \ &  \$-     & \  \\ \hline
	4.  ADP/Computer Services &  \$-    & \  &  \$-    & \  \\ \hline
	5.  Subawards/Consortium/Contractual Costs &  \$- & \   &  \$-      & \  \\ \hline
	6.  Equipment or Facility Rental/User Fees &  \$-   & \  &  \$-     & \  \\ \hline
	7.  Alterations and Renovations &  \$-  & \  &  \$-      & \  \\ \hline
	8.  Computer laptops (2) & \$4,000  & & \$1,800.41  & \  \\ \hline
	9. Oak Ridge National Laboratory &  \$40,000 &  &  \$40,000    & \  \\ \hline
	10. Other 3 &  \$-    & \ &  \$-      & \  \\ \hline
	\textit{Section G, Direct Costs (A thru F)} &  & \$161,409 & & \$128,043  \  \\ \hline
	\textit{Section H, Indirect Costs} &  & \$49,379 & & \$40,573 \  \\ \hline
	\textit{Section I, Total and Indirect Costs (G + H)} & &  \$210,788 &  &  \$168,616 \  \\ \hline
	Section J, Fee & \ &  \$-   & \ &  \$-       \\ \hline
\end{tabular}
\label{tab:costs}
\end{center}
\end{table}


\begin{table}
\begin{center}
\caption{Cost Differences Year 1}
\begin{tabular}{ | l | l | l | }
\hline
	\textbf{Item} & \textbf{Diff Per} & \textbf{Diff Total} \\ \hline
	\textit{Section A, Senior/Key Person} &  & \$6,786.88 \\ \hline
	\textit{Section B, Other Personnel} &  & \$18,272.81 \\ \hline
	Total Number Other Personnel &  & \  \\ \hline
	\textit{Total Salary, Wages and Fringe Benefits (A+B)} &  & \$25,059.69 \\ \hline
	\textit{Section C, Equipment} &  &  \$-    \\ \hline
	\textit{Section D, Travel} &  & \$5,561.76 \\ \hline
	1.  Domestic & \$5,561.76 & \  \\ \hline
	2.  Foreign &  \$-    & \  \\ \hline
	\textit{Section E, Participant/Trainee Support Costs} & \ &  \$-      \\ \hline
	1.  Tuition/Fees/Health Insurance &  \$-    & \  \\ \hline
	2.  Stipends &  \$-    & \  \\ \hline
	3.  Travel &  \$-    & \  \\ \hline
	4.  Subsistence &  \$-    & \  \\ \hline
	5.  Other &  \$-    & \  \\ \hline
	6.  Number of Participants/Trainees &  \$-    & \  \\ \hline
	\textit{Section F, Other Direct Costs} &  & \$2,744.59 \\ \hline
	1.  Materials and Supplies & \$45 & \  \\ \hline
	2.  Publication Costs & \$500 & \  \\ \hline
	3.  Consultant Services &  \$-    & \  \\ \hline
	4.  ADP/Computer Services &  \$-    & \  \\ \hline
	5.  Subawards/Consortium/Contractual Costs &  \$-    & \  \\ \hline
	6.  Equipment or Facility Rental/User Fees &  \$-    & \  \\ \hline
	7.  Alterations and Renovations &  \$-    & \  \\ \hline
	8.  Computer laptops (2) & \$2,199.59 & \  \\ \hline
	9. Oak Ridge National Laboratory &  \$-    & \  \\ \hline
	10. Other 3 &  \$-    & \  \\ \hline
	\textit{Section G, Direct Costs (A thru F)} &  & \$33,366.04 \\ \hline
	\textit{Section H, Indirect Costs} &  & \$8,805.46 \\ \hline
	\textit{Section I, Total and Indirect Costs (G + H)} &  & \$42,171.5 \\ \hline
	\textit{Section J, Fee }&  &  \$-    \\ \hline
\end{tabular}
\label{tab:differences}
\end{center}
\end{table}



\section{Schedule Status}
\label{sect::schedule}
Schedule Status.  List milestones, anticipated completion dates and actual completion dates.  If you submitted a project management plan with your application, you must use this plan to report schedule and budget variances.   

\section{Changes}
\label{sect::changes}
Describe any changes during the reporting period in project approach and the reasons for these changes.  Remember, significant changes to the project objectives and scope require prior approval by the Contracting Officer. 

\section{Anticipated Issues}
\label{sect::schedule}
6.	Describe any actual or anticipated problems or delays and any actions taken or plan to resolve them 

\section{Personnel Changes}
\label{sect::personnel}
Describe any absence or changes of key personnel or changes in consortium/teaming arrangements during the reporting period. 

\section{Products}
\label{sect::products}
List and describe any product produced or technology transfer activities accomplished during this reporting period, such as: \\
a)	Publications (list journal name, volume, issue); conference papers; or other public releases of results.  Attach or send copies of the public releases to the DOE Program Manager. \\
b)	Web site or other Internet sites (list the URL) that reflect the results of this project. \\
c)	Networks or collaborations fostered. \\
d)	Technologies/Techniques (identify and describe each). \\
e)	Inventions/Patent Applications (identify and describe with date of application) \\
f)	Other products, such as data or databases, physical collections, audio or video, software or NetWare, models, educational aid or curricula, instruments or equipment (identify and describe).

Other products: \href{https://github.com/munkm/caskmodels}{https://github.com/munkm/caskmodels}

%Publications: PHYSOR 2016 paper, ``FW/CADIS-$\Omega$: AN ANGLE-INFORMED HYBRID METHOD FOR DEEP-PENETRATION RADIATION TRANSPORT" can be found here\\ \href{http://munkm.github.io/papers/munk_physor16.pdf}{http://munkm.github.io/papers/munk\_physor16.pdf}


\bibliographystyle{physor2016}
\bibliography{year1-annual-report}

\appendix

\makeatletter
\def\@seccntformat#1{APPENDIX \csname the#1\endcsname.~}
\makeatother

%------------------------------------------------------------------------------
% If you need to make one (or more) appendix (appendices), place them here as
% sections
%%------------------------------------------------------------------------------
%\section{HOW TO MAKE APPENDICES}
%\label{app::a}
%
%This is a placeholder for my first appendix
%
%\section{OTHER APPENDIX STUFF}
%\label{app::b}
%
%This is a placeholder for my second appendix

\end{document}

