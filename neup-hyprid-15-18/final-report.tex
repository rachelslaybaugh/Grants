\documentclass[12pt]{article}

\usepackage{physor2016}

\usepackage{amsmath}
\usepackage{bm}
\usepackage{caption}
\usepackage{subcaption, paralist}
\usepackage{tabularx}
\usepackage{multirow}

\newcommand{\yr}{2}
\newcommand{\ve}[1]{\ensuremath{\mathbf{#1}}}
\newcommand{\Macro}{\ensuremath{\Sigma}}
\newcommand{\vOmega}{\ensuremath{\hat{\Omega}}}
\newcommand{\omvec}{\ensuremath{\hat{\Omega}}}
\newcommand{\rvec}{\ensuremath{\vec{r}}}
\newcommand{\vecr}{\ensuremath{\vec{r}}}
\newcommand{\co}{CADIS-$\Omega$ }


\pagestyle{myheadings}

\usepackage{graphicx}
\usepackage{tikz}
\usepgflibrary{shapes.geometric}
\usepackage{hyperref}
\hypersetup{
    colorlinks,
    citecolor=black,
    filecolor=black,
    linkcolor=black,
    urlcolor=black
}

\usepackage{booktabs}

\usepackage{siunitx}
%------------------------------------------------------------------------------

%------------------------------------------------------------------------------
% Define title. Use all CAPITALS.
%------------------------------------------------------------------------------
\title{Improved Hybrid Modeling of Spent Fuel Storage Facilities}
%\subtitle{}
%
% ...and authors
%
\author{ 
  Final Report for Project 14-6378\\
  January 2015 to December 2017\\
  \\
  \\
  Prepared by:\\
  Karl van Bibber (PI), Professor \\
  Department of Nuclear Engineering, University of California, Berkeley \\
  4153 Etcheverry Hall, Berkeley, CA 94720, USA\\
  \href{mailto:karl.van.bibber@berkeley.edu}{karl.van.bibber@berkeley.edu}\\
  \\
 Co-PIs: Rachel Slaybaugh, UC Berkeley\\
  Thomas Evans, Scott Moser; Oak Ridge National Laboratory\\
  \\
  \\
  Performed Under: NEUP Award DE-NE0008286\\
  Project period: 01/01/15 - 12/31/17 \\
  \\ 
  TPOC: Piyush Sabharwall\\
  FPOC: Daniel Vega\\
  \\
  \\
  Submitted March 21, 2018
}
  
%------------------------------------------------------------------------------
  

%------------------------------------------------------------------------------
% Setup PDF info. This sets several values which are listed as the "properties"
% of the PDF file.
%------------------------------------------------------------------------------
\hypersetup{
  pdftitle=\shorttitle,
  pdfauthor=\shortauthor
}


\begin{document}

%\doublespacing

%\linenumbers

%------------------------------------------------------------------------------
% Make the titlepage and set the pagestyle to fancy throughout
%------------------------------------------------------------------------------
\maketitle
\clearpage
\tableofcontents
\clearpage

%------------------------------------------------------------------------------
%
%------------------------------------------------------------------------------
\section{Executive Summary}
\label{sect::summary}

%which includes a discussion of 1) how the research adds to the understanding of the area investigated; 2) the technical effectiveness and economic feasibility of the methods or techniques investigated or demonstrated; or 3) how the project is otherwise of benefit to the public. The discussion should be a minimum of one paragraph and written in terms understandable by an educated layman.

This work developed a new computational method for improving the ability to calculate the neutron flux in deep-penetration radiation shielding problems that contain areas with strong streaming. The ``gold standard" method for radiation transport is Monte Carlo (MC) as it samples the physics exactly and requires few approximations. Historically, however, MC was not useful for shielding problems because of the computational challenge of following particles through dense shields. Instead, deterministic methods, which are superior in term of computational effort for these problems types but are not as accurate, were used.

Hybrid methods, which use deterministic solutions to improve MC calculations through a process called variance reduction, can make it tractable from a computational time and resource use perspective to use MC for deep-penetration shielding. Perhaps the most widespread and accessible of these methods are the Consistent Adjoint Driven Importance Sampling (CADIS) and Forward-Weighted CADIS (FW-CADIS) methods. For problems containing strong anisotropies, such as power plants with pipes through walls, spent fuel cask arrays, active interrogation, and locations with small air gaps or plates embedded in water or concrete, hybrid methods are still insufficiently accurate. 

In this work, a new method for generating variance reduction parameters for strongly anisotropic, deep-penetration radiation shielding studies was developed. This method generates an alternate form of the adjoint scalar flux quantity, $\phi^{\dagger}_{\Omega}$, which is used by both CADIS and FW-CADIS to generate variance reduction parameters for local and global response functions, respectively. The new method, called CADIS-$\Omega$, was implemented in the Denovo/ADVANTG software. Results indicate that the flux generated by CADIS-$\Omega$ incorporates localized angular anisotropies in the flux more effectively than standard methods. CADIS-$\Omega$ outperformed CADIS in several test problems. This initial work indicates that CADIS-$\Omega$ may be highly useful for shielding problems with strong angular anisotropies. This is a benefit to the public by increasing accuracy for lower computational effort for many problems that have energy, security, and economic importance. 

\clearpage


% ---------------------------------------------------
\section{Accomplishments}
\label{sect::accomplishments}
% Provide a comparison of the actual accomplishments with the goals and objectives of the project.

The main accomplishments in this project have been 
\begin{compactitem}
\item Completion of method implementation as well as testing and documentation. The method is available in software that can be obtained through RSICC.
\item All small test problems were constructed and executed with all versions of the software. 
\item Our large spent fuel shielding cask computational model was completed in two code formats and shared with ORNL for use.
\item Publication of our first initial results in a peer-reviewed conference, PHYSOR: \url{http://munkm.github.io/papers/munk\_physor16.pdf}
\item Established a set of anisotropy metrics and used them to quantify performance. 
\item Developed a suite of scripts to facilitate test execution and results plotting, as documented here: \url{https://github.com/munkm/thesiscode}.
\item Completion of the PhD of Dr.\ Madicken Munk \cite{Munk2017}.
\item Adoption of the developed method for use in other research projects.
\end{compactitem}

We accomplished nearly all of the goals outlined in the project. We have yet to complete the full-scale cask model analysis. Two students are conducting these calculations now. One journal article is in preparation that documents the results from Dr.\ Munk's dissertation. A second journal article will be published that includes the full cask model results. Thus, all work originally proposed in the project will be completed.



%------------------------------------------------------------------------------
%
%------------------------------------------------------------------------------
\section{Project Activities}
\label{sect::project}
%Summarize project activities for the entire period of funding, including original hypotheses, approaches used, problems encountered and departure from planned methodology, and an assessment of their impact on the project results. Include, if applicable, facts, figures, analyses, and assumptions used during the life of the project to support the conclusions.

In this project, we developed a variance reduction method for computational neutral particle transport to improve the ability to design and operate systems in which particle streaming is important, such as monitoring systems for interim used fuel installations. 
We implemented the method in Exnihilo \cite{evans_denovo:_2010}, MCNP \cite{brown_mcnp_2002}, and ADVANTG \cite{mosher_new_2010}, which makes our new tool widely available and easily usable. 
These innovative analysis tools will enable next generation nuclear material management for existing and future U.S.\ nuclear fuel cycles, minimizing proliferation and terrorism risk.

Storage casks are particularly challenging for radiation transport calculations because they are characterized by dense shields followed by streaming paths to the detectors. 
The impact of streaming is amplified in arrays of casks because detectors can only see rear casks through air paths between front casks. 
Deterministic methods suffer from ray effects between the casks and detectors, making solutions unreliable. 
Monte Carlo methods are challenged by getting particles out of the cask into the region of interest, and can therefore require unreasonably large calculation times to achieve acceptable statistical uncertainties in the computed tallies.

This report describes current technology in \autoref{sect::current} and the mathematics of our new method in \autoref{sect::omega}. This is followed by a short overview of the results and outcomes from this work in \autoref{sect::results}. Full details and all results can be found in Reference \cite{Munk2017}.

%------------------------------------
\subsection{Original Technology}
\label{sect::current}
Cutting-edge variance reduction methods that speed up Monte Carlo calculations often use deterministic solutions to make weight window maps. 
Perhaps the most widespread and accessible of these methods are the Consistent Adjoint Driven Importance Sampling (CADIS) \cite{wagner_automatic_1997,wagner_automated_1998,haghighat_monte_2003} and Forward-Weighted CADIS (FW-CADIS) \cite{wagner_forward-weighted_2007,wagner_forward-weighted_2009,wagner_forward-weighted_2010} methods. 
These methods create consistently-biased source distributions and weight window targets using a coarse determinstic solution for the adjoint scalar flux, $\phi^{\dagger}$, as a measure of the importance. 

In general, we are interested in finding some response, $R$, characterized by some response function $f(\ve{r}, E)$ in some volume $V_f$:
%
\begin{equation}
 R = \int_E \int_{V_f} f(\ve{r}, E) \phi(\ve{r}, E) dV dE \:,
 \label{eq:Response}
\end{equation}
where $\phi(\ve{r}, E)$ is the forward scalar flux, which describes how neutrons flow from the source $q(\ve{r}, E)$ to contribute to the response. 
Note that the adjoint scalar flux, $\phi^{\dagger}(\ve{r}, E)$, represents how each part of phase space contributes to the adjoint ``source" ($q^{\dagger}(\ve{r}, E)$, which can be set as the response of interest). 
Thus, $\phi^{\dagger}$ represents the expected contribution of a source particle to the desired response.
 
With this in mind, we can create variance reduction parameters for use in MC. 
We coarsely and quickly perform a deterministic calculation to get $\phi^{\dagger}(\ve{r}, E)$.
Equations \eqref{eq:CADISmethod} describe the biasing parameters generated by CADIS and FW-CADIS (jointly referred to as FW/CADIS). 
%
\begin{subequations} 
\label{eq:CADISmethod} 
\begin{equation}
\hat{q}(\vec {r} ,E)  = \frac{\phi^{\dagger}(\vec {r} ,E)q(\vec {r} ,E)}{\iint\phi^{\dagger}(\vec {r} ,E)q(\vec {r} ,E) dE d\vec{r}} \\
         = \frac{\phi^{\dagger}(\vec {r} ,E)q(\vec {r} ,E)}{R} \:,
\label{eq:weightedsource}
\end{equation}
\begin{equation}
w_0(\vec {r} ,E)  = \frac{q}{\hat{q}} \\
     = \frac{R}{\phi^{\dagger}(\vec {r} ,E)} \:,
\label{eq:startingweight}
\end{equation}
\begin{equation}
\hat{w}(\vec {r} ,E) = \frac{R}{\phi^{\dagger}(\vec {r} ,E)} \:,
\label{eq:WW}
\end{equation}
\end{subequations}
where $\hat{q}$ is the biased source distribution, $w_0$ is the starting weight of the particles, $\hat{w}$ is the target weight of the particles, and $R$ is the response of interest. 
For the standard implementations, these items are a function of space and energy only.

For problems with strong anisotropies in the particle flux, the importance map and biased source developed using the space/energy treatment above may not represent the real importance well enough to sufficiently improve performance in the Monte Carlo calculation. 
Note that because the \textit{scalar} adjoint flux is used in Eqns.~\eqref{eq:CADISmethod}, the angular dependence of the importance function is not retained. 
Thus, no information is retained on how particles move towards the response function. 
The drawback of this simplification is that, within a given space/energy cell, the map provides the average importance of a particle moving in \textit{any direction} through the cell\textemdash excluding information about how particles move \textit{toward} the objective. 
However, if the angular dependence of the importance function were fully retained, the map would be very large (tens or hundreds of GB) and more costly to use in the Monte Carlo simulation. 

An example of missing important angular behavior can be seen in this maze problem. 
A 10 MeV isotropic point source is next to a concrete maze followed by an NaI detector. 
This problem has vacuum boundary conditions. 
In \autoref{fig:maze} one can see the geometry and forward flux. 
\autoref{fig:orig-adj} shows the adjoint flux. 
We can see that (a) this is representing how areas of phase space contribute to the solution and (b) that no angular information is being captured.
In a vacuum system, particles that exit the back of the geometry should not affect the detector. 
This behavior is being missed by the standard method, and thus not speeding up MC as well as it could. 
\begin{figure}
\centering
\begin{subfigure}{.75\textwidth}
  \centering
  \includegraphics[height=2in,clip]{maze-forward.png}
  \caption{Forward Flux}
  \label{fig:maze}
\end{subfigure}%
\begin{subfigure}{.25\textwidth}
  \centering
  \includegraphics[height=1.1in,clip]{maze-adj-orig.png}
  \caption{Adjoint Close Up}
  \label{fig:orig-adj}
\end{subfigure}
\caption{Concrete Maze with 10 MeV isotropic point source and NaI detector}
\label{fig:adjoint}
\end{figure}

Interim used fuel installations exhibit strong angular anisotropies, and therefore the ability to simulate them effectively for nuclear material management is limited with current tools. 
This led us to develop a new method, which we're calling the FW/CADIS-$\Omega$ method. 
In MC without performance improvement, relative error (Re) decreases as the square of time (t). 
Thus, we measure improving a calculation by using the Figure of Merit (FOM):
\begin{equation}
\text{FOM}=\frac{1}{Re^{2}t}\:.
\end{equation}?

%---------------------------------
\subsection{CADIS-$\Omega$}
\label{sect::omega}
To do fast, accurate transport for used fuel monitoring, we need an importance map generated quickly using deterministic methods that captures the impact of angle in the importance information. 
In this work we build on past methods, but calculate the adjoint scalar flux in a way that has not been done before.

Our new automated hybrid method, which we're calling FW/CADIS-$\Omega$, incorporates angular information into the biasing parameters for FW/CADIS while not explicitly biasing in angle. 
That is, we generate space- and energy-dependent importance maps that incorporate the flux anisotropy in a more effective way than current implementations without adding the complication of angular weight windows. 
FW/CADIS-$\Omega$ uses Eqns.~\eqref{eq:CADISmethod}, but generates the adjoint scalar flux differently. 

We use the idea of the contributon flux, defined in Eqn.~\eqref{eq.Cont-Flux}, in generating the adjoint scalar flux. 
Contributons are pseudo-particles that carry ``response" from the radiation source to a detector ~\cite{williams_generalized_1991,williams_contributorn_1992,williams_contributon_study}. 
%
\begin{equation}
\Psi (\vec {r},\:\hat\Omega ,E) = \psi^{\dagger} (\vec {r},\:\hat\Omega ,E) \psi(\vec {r} ,\:\hat\Omega,E)
\label{eq.Cont-Flux} 
\end{equation}
%
The contributon flux includes both forward and adjoint information, expressing the importance of a particle that is born at a forward source and moves through space towards an adjoint source, contributing to the solution.
An importance map made using contributon flux will assign high importance to particles that are created at the forward source and are likely to generate a response in the detector. 

In FW/CADIS-$\Omega$, we integrate the contributon flux over angle and divide by the integrated forward angular flux as shown in Eqn.~\eqref{eq:angularhybrid}.
This quantity, which we designate $\phi^{\dagger}_{\Omega}$, is then used in Eqns.~\eqref{eq:CADISmethod}, just like the standard FW/CADIS methods.
%
\begin{equation} 
\phi^{\dagger}_{\Omega}(\vec{r},E) = \frac{\int_{4\pi} \psi(\vec {r} ,E,\hat{\Omega})\psi^{\dagger}(\vec {r} ,E,\hat{\Omega})d\hat\Omega }{\int_{4\pi}\psi(\vec {r} ,E,\hat{\Omega})d\hat\Omega}
\label{eq:angularhybrid}
\end{equation}

In a strongly anisotropic system, the adjoint scalar flux generated by Eqn.~\eqref{eq:angularhybrid} will be influenced by which directions were most prominent in the forward case. 
We can see this by considering the contributon flux, the numerator of Eqn.~\eqref{eq:angularhybrid}.
Particles in $\phi^{\dagger}_{\Omega}$ include the impact of how the direction they are moving influences the answer, which should allow for more effective Monte Carlo transport when angular effects are important. 
Note that in an isotropic system, $\phi^{\dagger}_{\Omega}$ will be essentially the same as $\phi^{\dagger}$. 

We can see the impact of this newly-defined adjoint flux by looking at the maze problem from \autoref{fig:adjoint}.
\autoref{fig:new-adj} shows a close up of the adjoint flux from our new method.
In this case it is clear that particles going out the back of the problem are not expected to contribute to the detector. This is the kind of result that makes sense, and demonstrates the appropriate incorporation of angular information.
Furthermore, we see in \autoref{fig:maze-re} that this is reflected in CADIS-$\Omega$ obtaining a lower relative error than either CADIS or analog MC for the same number of particles.
Finally, \co obtained a FOM of 145, while CADIS's was only 5.1. 
This illustrates the type of improvement this project can achieve. 
\begin{figure}
\centering
\begin{subfigure}{.25\textwidth}
  \centering
  \includegraphics[height=1.1in,clip]{maze-adj-new.png}
  \caption{New Method Adjoint Close Up}
  \label{fig:new-adj}
\end{subfigure}%
\begin{subfigure}{.75\textwidth}
  \centering
  \includegraphics[height=3in,clip]{maze-re.png}
  \caption{Relative Error for Analog, CADIS, and CADIS-$\Omega$}
  \label{fig:maze-re}
\end{subfigure}
\caption{Concrete Maze Comparison}
\label{fig:adjoint}
\end{figure}

One of the additionally useful items developed in this work is a collection of metrics to characterize anisotropy. 
Each metric can be used to help us characterize how our method preforms for what types of problems:
\vspace*{-.5em}
      \begin{itemize}
      \item Scalar Contributon Ratio: If the adjoint or forward angular flux is significantly peaked in $\vOmega$, this will result in a deviation between the numerator and denominator because there will be a multiplicative effect in the angular flux captured in $\Phi_{c}$ and not $\phi_{c}$.
      \[M_{1} = \frac{\phi^{\dagger}(\vecr,E)\phi(\vecr,E)}{\int_{\vOmega}\psi^{\dagger}(\vecr,\vOmega,E)\psi(\vecr,\vOmega,E)} = \frac{\phi_{c}}{\Phi_{c}}\]
      
      \item Adjoint Flux Ratio: metric for comparing which regions have significantly differing bias parameters in standard-adjoint and omega-adjoint situations. This metric will deviate from unity if the forward flux is anisotropic.
      \[M_{2} = \frac{\phi^{\dagger}_{\vOmega}(\vecr,E)}{\phi^{\dagger}(\vecr,E)}\]
      
      \item Maximum to Average Flux Ratios: the ratio between the maximum and average angular contributon flux in each space-energy voxel. The higher this quantity, the more peaked the contributon flux is in $\Omega$. 
      \begin{align*}
      \psi^{c} &= \psi^{\dagger}(\vecr,E, \vOmega)\psi(\vecr,E, \vOmega)\\
      M_{3} &= \frac{\psi^c_{\max}}{\psi^{c}_{\text{avg}}}\\
      M_{4} &=  \frac{\frac{\psi^{c}_{\max}}{\psi^{c}_{\text{avg}}}}{\frac{\psi^{\dagger}_{\max}}{\psi^{\dagger}_{\text{avg}}}} 
      \end{align*} 
      
       \item Maximum to Minimum Flux Ratios: this quantity incorporates information about the behavior of the local maximum relative to the local minimum angular flux in each cell. This metric may be more appropriate to describe the anisotropy of the flux in cells where the distribution of flux values are not well reflected by the average flux.
       \begin{align*}
      M_{5} &= \frac{\psi^c_{\max}}{\psi^{c}_{\min}}\\
      M_{4} &=  \frac{\frac{\psi^{c}_{\max}}{\psi^{c}_{\min}}}{\frac{\psi^{\dagger}_{\max}}{\psi^{\dagger}_{\min}}}
       \end{align*}
      \end{itemize} 
 
We used the Figure of Merit as well as these metrics to study a large number of test problems. 

% ---------------------------------------------------
\subsection{Results and Outcomes}
\label{sect::results}

The flux can have anisotropy resulting from more than one
mechanism. We have identified several separate
processes that affect the flux anisotropy. These processes can be grouped into
three categories that can overlap in one problem:
\begin{itemize}
  \itemsep0em
  \item anisotropy in the flux resulting from strongly directional sources,
  \item anisotropy resulting from strong differences between
material properties (this can be due to differences in
materials spatially or due to changes in interaction probabilities as a function
of energy),
  \item anisotropy in the flux from algorithmic limitations (ray effects).
\end{itemize}
There are four primary physical mechanisms by which the flux may
be anisotropic: streaming paths, problems with high
scattering effects, problems with high material heterogeneity (specifically with
materials with strong differences in scattering and absorption probabilities),
and problems with monodirectional sources. We created a set of test problems to investigate the new method's performance in all of these areas.

The labyrinth problems have isotropic point sources on the left hand side of the
problem emitting a Watt spectrum of
neutrons approximating the energy spectrum emitted by that of $^{235}$U fission.
On the right hand side of the problem there is a NaI detector recording the flux.
They are composed of a concrete maze with an air channel through the maze, and
then open air channels at either end of the channels. These problems are 
likely to have ray effects in the air region near the forward source. These problems have strong
differences in interaction probabilities between the air and the concrete,
thus they will have material heterogeneity. Further, because the concrete is
composed of several lighter-mass elements, these will also be highly scattering.


A steel beam embedded in
concrete with a NaI detector located on the right hand side of the problem was another challenge problem.
Because the particles have preferential flow through the steel and not the concrete, this is functionally a tough streaming problem. 

We also used a problem containing rebar in concrete with steel rebar running through
the concrete in different directions. In this problem, a NaI detector is
used to measure the response on the right hand side of the problem in yellow.
The source is both space- and energy-dependent, emitting a Watt spectrum of
neutrons characteristic of $^{235}$U fission, and is distributed in a 100x160
centimeter plate
on the left hand side of the problem. The source is monodirectional in $+x$.
This problem will have angular
dependence, but preferential flowpaths through the concrete are not directed
towards the detector location on the other side of the shielding in some of the
rebar. This problem has material heterogeneity both in the concrete and between
the concrete and air. This problem is highly scattering from the concrete, and
is unlikely to have ray effects without a strong single preferential flowpath
through the shield.

Finally, a small application problem relevant to the interests of this project is a
radiotherapy room that has concrete
walls, a water-based phantom that is being irradiated by a monodirectional
source in the room, and a hallway where a therapy technician might walk.  Because this problem is primarily air with concrete
borders, it will have strong streaming effects in the air. 
Because of the high fraction of air in this problem, we also anticipate
ray effects to occur. While there will be scattering in this problem, it will
not be as strong of an effect as other characterization problems.

The performance of CADIS-$\Omega$ was characterized and compared against
CADIS and a standard, nonbiased analog Monte Carlo run for a series of problems.
We found that CADIS-$\Omega$ does not outperform
CADIS for all problems containing anisotropy in the flux. Depending on how and
where the flux
anisotropy was induced in the problem, CADIS-$\Omega$ had the potential to
significantly increase the FOM in Monte Carlo. These results were not
consistent, and are not entirely predictable.

In comparing the single turn and multiple turn labyrinths, it was observed that
more scattering effects decrease the effectiveness of CADIS-$\Omega$. Because
more scattering is required to penetrate the multiple turn labyrinth,
the performance of CADIS-$\Omega$ was poorer. In the single turn labyrinth
energy bins that had more isotropy in the flux induced by scattering also were
poorer performing for CADIS-$\Omega$.

To add to this complexity, problems with little or no scattering were also
difficult for CADIS-$\Omega$ to handle. These problems were also problematic for
CADIS, as they were generally comprised of ``thin'' materials to induce
streaming effects. As a result, sampling events occurred over several
centimeters, which also was over several orders of magnitude in flux change.
This resulted in very high relative errors, as observed in the beam facility
problem. This was not as problematic in the therapy room example because the
problem was bounded by 10 cm of concrete, which allowed for particle scattering
rather than leakage.

Several material variants of the steel beam in concrete problem were run. The
results of this small study confirmed that both CADIS and CADIS-$\Omega$ obtain
poorer FOMs with air than with steel or concrete. In the case of the air
variant, the FOMs obtained by CADIS-$\Omega$ were generally lower than CADIS,
but the relative errors were also better. For all material variants of the steel
beam problem, CADIS and CADIS-$\Omega$ achieved superior FOMs to the nonbiased
analog, but these were an order of magnitude lower for the air variant.

The rebar-embedded concrete problem showed that for problems with
geometric complexity, CADIS-$\Omega$ can also struggle. Because the rebar in
this problem was not always directed in line with the detector tally, particles
could more freely move perpendicular to the tally path, crossing out of
importance with a preferential flowpath. As a result, in high energy bins the
tally relative error was very high for both CADIS and CADIS-$\Omega$. However,
CADIS-$\Omega$'s performance was poorer. The FOMs obtained by CADIS-$\Omega$ in
this problem were one to two orders of magnitude smaller than CADIS or the
nonbiased analog.

CADIS-$\Omega$ achieved lower relative errors than CADIS for many problems, but
often this was offset by a very long runtime. The long runtime impacted the FOM.
As a result, even though CADIS-$\Omega$ achieves a lower relative error for the
same particle count, it may be more advantageous to simply run standard CADIS
for longer. In a few instances, the runtime for CADIS-$\Omega$ is comparable to
CADIS. This occurs in the beam and therapy room problems, for example. Although
these problems are not the best for either CADIS or CADIS-$\Omega$, there is no
caveat to using CADIS-$\Omega$ if choosing a hybrid method.

The characterization problems' variations in material and geometric
configuration showed that there is no distinct behavior for which CADIS-$\Omega$
is universally better. However,
in problem geometries where preferential flowpaths are directed
towards the tally detector, and where materials provide short mean free paths to
interaction or resampling sites, CADIS-$\Omega$ is a well-suited method.

The angle-based parametric study provided a number of interesting observations on
the performance of the $\Omega$ methods.
First, the effect of T$_{det}$ does not change the FOM with CADIS-$\Omega$ more
than CADIS. The hypothesis that I/O requirements would severely
impact the FOM for CADIS-$\Omega$ was shown to be lower than
hypothesized. The FOM change between
FOM$_{MC}$ and FOM$_{hybrid}$ was roughly the same for CADIS as CADIS-$\Omega$
because the CADIS-$\Omega$ runtimes are so much longer than CADIS.

Next, the only consistent region in which CADIS-$\Omega$ outperforms CADIS is in
high energies. For almost all P$_N$ orders and all quadrature orders, CADIS-$\Omega$
achieved lower relative errors and higher FOMs than CADIS. In high energy bins,
increasing quadrature order showed a decrease in I$_RE$, increasing P$_N$ order
did not show a large change in I$_RE$. In the same bins, I$_{FOM}$ values above
unity were observed for both P$_N$ and S$_N$ order, but no trends with changing
parameter value were observed.

By including the runtime to calculate
the FOM, the comparative performance of CADIS-$\Omega$ dropped when compared to
using the relative error. Several
energy bins in CADIS-$\Omega$--for quadrature orders and
P$_N$ orders--achieved better FOMs than
CADIS. However no P$_N$ order consistently outperformed the other, while low
S$_N$ orders generally achieved better FOMS for CADIS-$\Omega$ than CADIS.
However, despite the lack of consistent performance for a single P$_N$ order,
the raw values obtained with P$_N$ order are promising. With P$_N$ order
there were more energy bins that had high I$_{FOM}$ values than with quadrature
order.

Another observation was
that CADIS-$\Omega$ consistently biases
particles better than CADIS. For the same number of source particles,
CADIS-$\Omega$
achieves lower relative error than CADIS for most energy bins with both P$_N$
order and quadrature order. This means that while sampling may be slow, the
importance map generated with the $\Omega$ flux is generally
better at moving particles to the tally region than
CADIS.

Based on the results, a number of
recommendations can be made based on deterministic solver choice. First, the
best P$_N$ order choice is
dependent on the energy range in which one is tallying. For low energy regions,
P$_N$ order 1 will give the best FOMs relative to CADIS, for intermediate
energies P$_N$ 3 is a better choice, and for high energies any P$_N$ order is
satisfactory. In general, because lower P$_N$ orders have lower runtimes, these
will get the best results for CADIS-$\Omega$ the fastest, and have comparatively
the best relative errors and FOMs against CADIS. Next, the best S$_N$ order
choice is

If one has to choose between varying P$_N$ order and S$_N$ order to improve the
importance map for their method, varying S$_N$ order will have a greater impact.
This is the case for using either CADIS or CADIS-$\Omega$. However, both methods
have a turnarount point at which increasing S$_N$ order does not improve the
relative error enough to offset the time increase of the method. For
CADIS-$\Omega$, this occurs in bins above S$_N$ 15, and for CADIS it occurs in
bins aboe S$_N$ 12. For this type of problem, and using all energy bins in the
tally, CADIS-$\Omega$ will obtain the
best results with a lower P$_N$ order and intermediate S$_N$ orders.

The characterization problems that were run were heavily biased towards
low-density streaming to induce anisotropy in the flux. This subset of
problems, though highly anisotropic, are not the best
for a method so dependent on weight-window type biasing,
because particle streaming allowed for particles to cross several orders of
magnitude in the flux before re-sampling. This meant that in a high-importance
region a particle may split many thousands of times in a new splitting event.
Unfortunately, the $\Omega$-methods are not immune to this issue and so suffered
the same effects as CADIS, even with positive effects like the reduction of ray
effects. Further, with the strong dependence on angle, the $\Omega$-fluxes may
have exacerbated this streaming-sampling effect in regions with strong angular
dependence around the detector. In a problem like the single turn labyrinth,
where the $\Omega$-flux generated a strong line of importance between the exit
of the labyrinth and the detector and drastically dropped the importance behind
the detector, a particle has much more opportunity to cross several orders of
magnitude of importance than it does in CADIS. This is likely what caused
CADIS-$\Omega$ to take longer in Monte Carlo transport than CADIS in many of the
characterization problems.

It should also be noted that while the angle-dependent parametric study revealed
how P$_N$ order and quadrature order may affect a problem's results, the
best parameter choices for this problem are by no means a prescriptive solution
for other problems. Different the characterization problems' results were, depending on
the source definition, the material composition of the problem, and the
geometric configuration of the problem. Using the deterministic parameter
choices that appear the best for the steel beam in concrete may not be the best
for, say, a multi-turn labyrinth. From this study we have a good starting point
from which to further characterize the method for other application problems.




% ---------------------------------------------------
% ---------------------------------------------------
% ---------------------------------------------------
\section{Products}
\label{sect::products}
%List and describe any product produced or technology transfer activities accomplished during this reporting period, such as: \\
%Publications (list journal name, volume, issue); conference papers; or other public releases of results.  Attach or send copies of the public releases to the DOE Program Manager. \\
\textit{Publications:} 
\begin{compactitem}
\item PHYSOR 2016 paper, ``FW/CADIS-$\Omega$: AN ANGLE-INFORMED HYBRID METHOD FOR DEEP-PENETRATION RADIATION TRANSPORT" can be found at \url{http://munkm.github.io/papers/munk\_physor16.pdf}

\item Madicken Munk's dissertation, ``FW/CADIS-$\Omega$: An Angle-Informed Hybrid Method for Neutron Transport", can be found at \url{http://github.com/munkm/dissertation/thesis.pdf} \cite{Munk2017}
\end{compactitem}

%b)	Web site or other Internet sites (list the URL) that reflect the results of this project. \\
\textit{Website:} 
\begin{compactitem}
\item The GitHub respository that contains code use information, brainstorming, publicly-available details about the cask, process development, and citations \url{https://github.com/munkm/caskmodels}

\item We also have a repository with scripts and plotting tools: \url{https://github.com/munkm/thesiscode}
\end{compactitem}

%c)	Networks or collaborations fostered. \\
\textit{Networks or collaborations:} 
\begin{compactitem}
\item We have grown the collaboration between ORNL and UCB, with Dr.\ Munk spending a few months at ORNL learning the codes and building our network. 

\item The work that we did in this project has formed a tool being used in a new project funded by the DOE-NE NEUP program.  The new project is studying reprocessing facilities and is formally in collaboration with Southern Company and informally with Sandia National Laboratory. 

\item Ms.\ Vanessa Goss and Ms.\ Emily Vu are continuing analysis work with the software developed in this project. Ms.\ Goss will spend the summer at ORNL; Ms.\ Vu will spend the summer at SNL.

\item Ms.\ Kelly Rowland is completing a PhD developing a related method that is useful for similar problem types. 
\end{compactitem}

%d)	Technologies/Techniques (identify and describe each). \\
\textit{Technologies or Techniques:} As described in the proposal, the new method is our main technique. 
%e)	Inventions/Patent Applications (identify and describe with date of application) \\

%f)	Other products, such as data or databases, physical collections, audio or video, software or NetWare, models, educational aid or curricula, instruments or equipment (identify and describe).
\textit{Other products:} 
\begin{compactitem}
\item The full cask model we created in both SCALE and MCNP formats has been contributed back to ORNL for use by the wider community.

\item The new method has been implemented in software (Exnihilo / ADVANTG) that is available through RSICC, so this product is also available to others. 
\end{compactitem}


% ---------------------------------------------------
\section{Computer Modeling}
\label{sect::modeling}

The details related to computer modeling are included in the Project Activities description and the referenced publications:
\begin{compactitem}
\item Model description, key assumptions, version, source and intended use;
\item Performance criteria for the model related to the intended use;
\item Test results to demonstrate the model performance criteria were met (e.g., code
verification/validation, sensitivity analyses, history matching with lab or field data, as
appropriate);
\item Theory behind the model, expressed in non‐mathematical terms;
\item Mathematics to be used, including formulas and calculation methods;
\item Whether or not the theory and mathematical algorithms were peer reviewed, and, if so,
include a summary of theoretical strengths and weaknesses;
\item Hardware requirements; and
\item Documentation (e.g., user guide, model code).
\end{compactitem}


\bibliographystyle{physor2016}
\bibliography{year1-annual-report}

\appendix

\makeatletter
\def\@seccntformat#1{APPENDIX \csname the#1\endcsname.~}
\makeatother

%------------------------------------------------------------------------------
% If you need to make one (or more) appendix (appendices), place them here as
% sections
%%------------------------------------------------------------------------------
%\section{HOW TO MAKE APPENDICES}
%\label{app::a}
%
%This is a placeholder for my first appendix
%
%\section{OTHER APPENDIX STUFF}
%\label{app::b}
%
%This is a placeholder for my second appendix

\end{document}

