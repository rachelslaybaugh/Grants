\documentclass{proposalnsf}
% This class file has been tweaked to death by LBarba to fit precisely the 
% formatting strictures of NSF, while still being rather pretty.

%--------------------------------------------------------------------  PROCESS WITH XeLaTeX
%\usepackage{fontspec}% provides font selecting commands 
%\usepackage{paralist}       % compactitem environment
%\usepackage{xunicode}% provides unicode character macros 
%\usepackage{xltxtra} % provides some fixes/extras 
%\setromanfont[Mapping=tex-text,
%                 SmallCapsFont={Palatino},
%                 SmallCapsFeatures={Scale=0.85}]{Palatino}
%\setsansfont[Scale=0.85]{Trebuchet MS} 
%\setmonofont[Scale=0.85]{Monaco}

\renewcommand{\captionlabelfont}{\bf\sffamily}
\usepackage[hang,flushmargin]{footmisc} 
% 'hang' flushes the footnote marker to the left,  'flushmargin'  flushes the text as well.

% Define the color to use in links:
\definecolor{linkcol}{rgb}{0.459,0.071,0.294}
\definecolor{sectcol}{rgb}{0.63,0.16,0.16} % {0,0,0}
\definecolor{propcol}{rgb}{0.75,0.0,0.04}

\definecolor{gray}{rgb}{0.25,0.25,0.25}
\definecolor{ngreen}{rgb}{0.7,0.7,0.7} % a darker shade of green

\usepackage[
    %xetex,
    pdftitle={Packard Fellowship Research Statement},
    pdfauthor={Rachel Slaybaugh},
    pdfpagemode={UseOutlines},
    pdfpagelayout={TwoColumnRight},
    bookmarks, bookmarksopen,bookmarksnumbered={True},
    pdfstartview={FitH},
    colorlinks, linkcolor={sectcol},citecolor={sectcol},urlcolor={sectcol}
    ]{hyperref}

%% Define a new style for the url package that will use a smaller font.
\makeatletter
\def\url@leostyle{%
  \@ifundefined{selectfont}{\def\UrlFont{\sf}}{\def\UrlFont{\small\ttfamily}}}
\makeatother
%% Now actually use the newly defined style.
\urlstyle{leo}


% this handles hanging indents for publications
\def\rrr#1\\{\par
\medskip\hbox{\vbox{\parindent=2em\hsize=6.12in
\hangindent=4em\hangafter=1#1}}}


\addto\captionsamerican{%
  \renewcommand{\refname}%
    {References Cited}%
} % solution found here: http://www.tex.ac.uk/cgi-bin/texfaq2html?label=latexwords

\def\baselinestretch{1}
\setlength{\parindent}{0mm} \setlength{\parskip}{0.8em}

\newlength{\up}
\setlength{\up}{-4mm}

\newlength{\hup}
\setlength{\hup}{-2mm}

\sectionfont{\large\bfseries\color{sectcol}\vspace{-2mm}}
\subsectionfont{\normalsize\it\bfseries\vspace{-4mm}}
\subsubsectionfont{\normalsize\mdseries\itshape\vspace{-4mm}} %\itshape
\paragraphfont{\bfseries}

% ---------------------------------------------------------------------
% DRAFTING COMMENTS:
\newcommand\ignore[1]{} % Styles for author comments:
% Enter a comment like this:   \comment{This is a comment.}
%\ignore{
\newcommand{\important}[1]{\textcolor{red}{ #1 }}
\newcommand{\comment}[1]{\textcolor{blue}{ #1 }}
%}
% Uncomment lines below to change from visible to invisible comments.
%\renewcommand{\important}[1]{}
%\renewcommand{\comment}[1]{}

\begin{document}
% Small title with normal-size font:
\begin{center}

\textbf{Pakard Fellowship Research Statement}\\
Rachel N.\ Slaybaugh, Assistant Professor\\
Department of Nuclear Engineering, University of California, Berkeley

\end{center}

MC on advanced architectures. Big challenge in current climate; building machines
at the extreme scale and we're working on that. No one is really actually trying
to take this to the vendor/industry scale to change how we do things. 

Basis: we have a code that runs on 1 GPU and 1 CPU, but it's very early. We 
will really need to work to investigate methods to get this to perform better on
that system. We further need to expand this to other architecture types so that
we aren't tied only to GPUs (e.g.\ MICs) and allow for multi-CPU + GPU systems.

I'm on a propsal to get early machine access. This will provide access to the 
architecture, the hardware vendors, and the compiler and language developers.
However, this award does not provide funding for me. 

Additional component: I'd really like to be able to have the resources to 
connect with industry, expecially young employees who are more open to new 
methods and ideas. This could help fund exchanges with them to really bring
their problems into my research.

5 year timeline:
\begin{itemize}
\item Year 1:
    \begin{itemize}
    \item Investigate expanding WARP to MICs
    \item Complete delta-tracking geometry investigation
    \item Extend WARP to multi-CPU
    \end{itemize}

\item Year 2:
    \begin{itemize}
    \item Design integration framework for generic architecture strategy
    \item Tackle strategies for isotope handling
    \item Begin working with NuScale and TP so they can try out the code
    \end{itemize}

\item Year 3:
    \begin{itemize}
    \item Implement and test framework for generic architecture (if possible)
    \item Investigate VR methods that are targeted for GPUs and MICs; look for
          innovative plans for overlap
    \item Use TP and NuScale feedback for code design modifications
    \end{itemize}

\item Year 4:
    \begin{itemize}
    \item Continue development and testing of VR methods
    \item Begin investigating best methods for multi-physics integration
    \item TP and NuScale design calculation tests
    \end{itemize}

\item Year 5:
    \begin{itemize}
    \item Continue testing and implementation of multi-physics integration
    \item Take code to large vendors (Westinghouse, Areva) for testing and input
    \end{itemize}
\end{itemize}

Other ideas: 
\begin{itemize}
\item Multigroup cross section processing combined with nuclear data understanding
      (this is much more outside of my experience, but I think it's really 
       important)
\item Use PyNE to build plug-and-play research toolbox. Allows for rapid
      investigation of new numerical methods. The buildup will be slow, but if 
      well thought out I think it could actually be agile. This also relies on 
      me having ideas - but I still have some about eval solvers, preconditioners,
      etc. Could also be a framework for code coupling?
\end{itemize}


The Fellowship Program provides support for highly
creative researchers early in their careers; faculty
members who are well established and well‐funded
are less likely to receive the award.  Packard Fellows
are inquisitive, passionate scientists and engineers
who take a creative approach to their research, dare
to think big, and follow new ideas wherever they lead.  
The Foundation emphasizes support for innovative
individual research that involves the Fellows, their
students, and junior colleagues, rather than
extensions or components of large‐scale, ongoing
research programs.  

The research statement should describe why
the research is important and outline the
general goals for the next five years. The
statement should also indicate, in general,
how funds will be used. This does not need to
be a detailed budget and will not be binding
on the actual use of funds. The research
statement is limited to 1,400 words
(maximum two pages of text, prepared in 12‐
point font with 1‐inch margins). If there are
relevant figures, images or references, please
include these separately on a third page.  

\end{document}
